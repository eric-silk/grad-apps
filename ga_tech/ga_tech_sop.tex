% !TEX program = xelatex
%%%%%%%%%%%%%%%%%%%%%%%%%%%%%%%%%%%%%%%%%%%%%%%%%%%%%%%%%%%%%%%%%%%%%
%% Title: SOP LaTeX Template
%% Author: Soonho Kong / soonhok@cs.cmu.edu
%% Modified by: Eric Silk
%% Created: 2012-11-12
%% Modified: 2021
%%%%%%%%%%%%%%%%%%%%%%%%%%%%%%%%%%%%%%%%%%%%%%%%%%%%%%%%%%%%%%%%%%%%%

%%%%%%%%%%%%%%%%%%%%%%%%%%%%%%%%%%%%%%%%%%%%%%%%%%%%%%%%%%%%%%%%%%%%%
%%
%% Requirement:
%%     You need to have the `Adobe Caslon Pro` font family.
%%     For more information, please visit:
%%     http://store1.adobe.com/cfusion/store/html/index.cfm?store=OLS-US&event=displayFontPackage&code=1712
%%
%% How to Compile:
%%     $ xelatex main.tex
%%
%%%%%%%%%%%%%%%%%%%%%%%%%%%%%%%%%%%%%%%%%%%%%%%%%%%%%%%%%%%%%%%%%%%%%

\documentclass[letterpaper]{article}
\usepackage[letterpaper,margin=1.0in,noheadfoot]{geometry}
\usepackage{fontspec, color, enumerate, sectsty}
\usepackage[normalem]{ulem}

%%%%%%%%%%%%%%%%%%%%%%%%%%%%%%%%%%%%%%%%%%%%%%%%%%%%%%%%%%%%%%%%%%%%%
%                      YOUR INFORMATION
%
%      PLEASE EDIT THE FOLLOWING LINES ACCORDINGLY!!
%%%%%%%%%%%%%%%%%%%%%%%%%%%%%%%%%%%%%%%%%%%%%%%%%%%%%%%%%%%%%%%%%%%%%
\newcommand{\soptitle}{Statement of Purpose}
\newcommand{\yourname}{Eric Silk}
\newcommand{\youremail}{esilk16@uw.edu}

%% FONTS SETUP
\defaultfontfeatures{Mapping=tex-text}
\usepackage[bookmarks, colorlinks, breaklinks,
pdftitle={\yourname - \soptitle},pdfauthor={\yourname}, unicode]{hyperref}
\hypersetup{linkcolor=magneta,citecolor=magenta,filecolor=magenta,urlcolor=[named]{WildStrawberry}}

%%%%%%%%%%%%%%%%%%%%%%%%%%%%%%%%%%%%%%%%%%%%%%%%%%%%%%%%%%%%%%%%%%%%%
%                      Title and Author Name
%%%%%%%%%%%%%%%%%%%%%%%%%%%%%%%%%%%%%%%%%%%%%%%%%%%%%%%%%%%%%%%%%%%%%
\begin{document}
\begin{center}{\huge \scshape \soptitle}\end{center}
\begin{center}\vspace{0.2em} {\Large \yourname\\}
  {\youremail}\end{center}
\frenchspacing



%%%%%%%%%%%%%%%%%%%%%%%%%%%%%%%%%%%%%%%%%%%%%%%%%%%%%%%%%%%%%%%%%%%%%
%                      SOP Body
% NOTE: Use \amper instead of \&
%%%%%%%%%%%%%%%%%%%%%%%%%%%%%%%%%%%%%%%%%%%%%%%%%%%%%%%%%%%%%%%%%%%%%
\section*{Motivation and Background}
I am a Research Engineer at Schweitzer Engineering Laboratories, Inc., and graduating masters
student of Applied Mathematics at the University of Washington with a Bachelors of Science
in Electrical Engineering from the University of Idaho. My employer's mission is
one I believe in: ``To make electric power safer, more reliable, and more economical.'' On a daily
basis I perform at the intersection of applied math, electrical engineering, and software
engineering to further this goal.

Upon transferring to SEL's Government Services division from R\&D, it became apparent
there is a need for people who are capable of bridging the gap between the abstract
and esoteric world of mathematics and real-world applications in engineering.
Thus, I elected to pursue graduate education.
Even with the huge strides I have made, my work has just started. Earning a PhD will allow me to
transition into an independent, productive researcher and to speak with authority on topics
related to electric power.

\section*{Relevant Experience}
\paragraph{Career Research}
Due to the nature of my work for my employer, I am not allowed to discuss many specifics
and definitely not allowed to publish. We are recognized as experts
on power systems by various federal agencies and provide research, analysis,
and product in this space. In particular, the Data Analytics group marries this expertise
with high performance computing, machine learning, and traditional applied mathematics to
achieve novel and challenging goals. Of what I can discuss, the following stand out.

In February, I submitted a patent application for a compression method
to help streamline the massive data output of SEL's flagship relay, the T400L/T401L --- an
18-bit, megahertz sampling relay for the detection of traveling wave events.
We have a vested interest in long-term data captures at this
fidelity, but the existing implementation had excessive storage requirements
(four terabytes for three weeks). During Dr. Kutz's class (AMATH 582), I applied his theme of
assuming structure to solve $Ax=b$ given only $b$ and
assume an autoregressive model and produced lossy compression. Then, the errors are stored in
Golomb-Rice codes, producing a reduction of 5.7 to 7.4 times less than prior art.
This was a major achievement --- it enabled storing one year of data on a single 3.5"
drive, or streaming using less bandwidth than Netflix.

My first assignment in Government Services was developing sensor system to measure and subsequently
store Synchrophasor data (GPS timestamped frequency, voltage, and phase) on the cloud. This involved
selecting a computing platform, developing code to interact with a phasor measurement unit,
robust logging and system restarts, and code for Amazon Web Services' IoT,
Lambda, and S3 services -- all while ensuring the devices were secure. We successfully deployed
several devices that have now performed uninterrupted for several years.

A recent achievement was the synthesis of a massive schematic dataset for bootstrapping a machine
learning model. Over ten-thousand Eagle schematics were generated, including
resistors, capacitors, transistors, integrated circuits, and more.
This included bounding boxes and component identifiers --- all without human intervention. I am
actively expanding this capability to include more CAD programs.

Finally, I have applied optimization in several influential ways. The first was generating
nonlinear transmission lines via genetic algorithms to achieve particular performance
characteristics that would be intractable with traditional engineering techniques.
Another project involved fitting high-order, nonlinear transformer parameters through genetic
algorithms. This was quite successful, taught me that judicious use of penalties prevents
computationally correct but physically nonsensical models, and is still in use by our power
engineering team. Finally, I implemented a crest-factor minimization problem in PyTorch that will
serve as a testbed for different optimizers, and introduced autodifferentiation and quasi-newton
methods into our repertoire.


\paragraph{Independent Study and Scalable Second Order Optimizers}
As an online Masters student, I was unable to write a thesis. To ameliorate this, I arranged an
independent study course with Dr. Andrew Lumsdaine.
Topics ranged from power system simulations using PETSc, matrix decompositions in CUDA, and the
intersection of graphs and linear algebra. I learned an incredible amount and made enough of an
impression to help with his Second Order Methods for Scalable Optimization project. This project
hinges around the use of finite-difference Hessian approximations, along with limited-memory
techniques to accelerate machine learning training. Our git repository can be found here:
https://github.com/lums658/ml20

\section*{Career Goals}
I foresee three career trajectories: professorship, industry, and civil service.
All share a common goal: perform at the intersection of applied math and engineering
and provide relevant, timely, and digestible research and education to students,
peers, or customers.

As an educator, I would provide power engineers with the tools
to step beyond traditional power engineering analyses and engage with cutting-edge applied
mathematics techniques, while providing context and constraints to keep these techniques grounded in
industrial realities.  In industry or civil service, I would continue in much my same role:
exemplary work for government customers making the grid safer, cheaper, and more reliable.
This would take the form of consulting and providing tools for use by the federal
government, and sound advice for shaping policy.

\section*{Challenges}
I want to address two points. First, I had a medical withdrawal in undergrad with lingering
effects in subsequent semesters. This issue has since been fully resolved, as evidenced
by my final semesters and my graduate performance. Secondly, my withdrawal in graduate school was
due to a death in my family. Neither pose risks to future academic endeavors.

\section*{Institution Selection}
When discussing PhD programs that would straddle applied math and electrical engineering, my
advisor's first words were "Georgia Tech". Your program has produced several professors at other
programs I am applying to, and is one of the few to explicitly sit in the space I want to perform:
rigorous mathematics to develop clever algorithms for effective engineering.

\section*{Concluding Remarks}
I hope I've demonstrated a clear upward trajectory in my career and research.
I'm on the cusp of being a strong and independent researcher; a PhD from your program
would be the keystone in my ability to contribute to electrical engineering and applied
mathematics.
\end{document}
