% !TEX program = xelatex
%%%%%%%%%%%%%%%%%%%%%%%%%%%%%%%%%%%%%%%%%%%%%%%%%%%%%%%%%%%%%%%%%%%%%
%% Title: SOP LaTeX Template
%% Author: Soonho Kong / soonhok@cs.cmu.edu
%% Modified by: Eric Silk
%% Created: 2012-11-12
%% Modified: 2021
%%%%%%%%%%%%%%%%%%%%%%%%%%%%%%%%%%%%%%%%%%%%%%%%%%%%%%%%%%%%%%%%%%%%%

%%%%%%%%%%%%%%%%%%%%%%%%%%%%%%%%%%%%%%%%%%%%%%%%%%%%%%%%%%%%%%%%%%%%%
%%
%% Requirement:
%%     You need to have the `Adobe Caslon Pro` font family.
%%     For more information, please visit:
%%     http://store1.adobe.com/cfusion/store/html/index.cfm?store=OLS-US&event=displayFontPackage&code=1712
%%
%% How to Compile:
%%     $ xelatex main.tex
%%
%%%%%%%%%%%%%%%%%%%%%%%%%%%%%%%%%%%%%%%%%%%%%%%%%%%%%%%%%%%%%%%%%%%%%

\documentclass[letterpaper]{article}
\usepackage[letterpaper,margin=1.0in,noheadfoot]{geometry}
\usepackage{fontspec, color, enumerate, sectsty}
\usepackage[normalem]{ulem}

%%%%%%%%%%%%%%%%%%%%%%%%%%%%%%%%%%%%%%%%%%%%%%%%%%%%%%%%%%%%%%%%%%%%%
%                      YOUR INFORMATION
%
%      PLEASE EDIT THE FOLLOWING LINES ACCORDINGLY!!
%%%%%%%%%%%%%%%%%%%%%%%%%%%%%%%%%%%%%%%%%%%%%%%%%%%%%%%%%%%%%%%%%%%%%
\newcommand{\soptitle}{Personal Statement}
\newcommand{\yourname}{Eric Silk}
\newcommand{\youremail}{esilk16@uw.edu}

%% FONTS SETUP
\defaultfontfeatures{Mapping=tex-text}
\usepackage[bookmarks, colorlinks, breaklinks,
pdftitle={\yourname - \soptitle},pdfauthor={\yourname}, unicode]{hyperref}
\hypersetup{linkcolor=magneta,citecolor=magenta,filecolor=magenta,urlcolor=[named]{WildStrawberry}}

%%%%%%%%%%%%%%%%%%%%%%%%%%%%%%%%%%%%%%%%%%%%%%%%%%%%%%%%%%%%%%%%%%%%%
%                      Title and Author Name
%%%%%%%%%%%%%%%%%%%%%%%%%%%%%%%%%%%%%%%%%%%%%%%%%%%%%%%%%%%%%%%%%%%%%
\begin{document}
\begin{center}{\huge \scshape \soptitle}\end{center}
\begin{center}\vspace{0.2em} {\Large \yourname\\}
  {\youremail}\end{center}
\frenchspacing



%%%%%%%%%%%%%%%%%%%%%%%%%%%%%%%%%%%%%%%%%%%%%%%%%%%%%%%%%%%%%%%%%%%%%
%                      SOP Body
% NOTE: Use \amper instead of \&
%%%%%%%%%%%%%%%%%%%%%%%%%%%%%%%%%%%%%%%%%%%%%%%%%%%%%%%%%%%%%%%%%%%%%
\section*{Asking the Right Questions}
Johari windows were originally developed to
understand interpersonal relationships and were famously brought into the mainstream
by Donald Rumsfeld's statement of ``unknown unknowns.'' Succinctly, every person and
every field of study has at least two ``blind spots'' --- that which we know we do not
know, and that which we don't know we don't know. This problem manifests itself in two ways
critical to research and academia: the readily apparent problem posed by research itself,
and the meta-problem of how we produce an environment that can solve it. My graduate education
was motivated by the desire to tackle these problems.

\subsection*{Research}
A coworker of mine was working on a problem in optimization called crest factor minimization: for
a given series of sinusoids of fixed amplitude and frequency, seeks to minimize the ratio of peak
value to RMS value (the infinity norm over the 2-norm). In its formulation were three ``funny bits.''
The first, a unique optimizer called Levenberg-Marquardt, which he was unfamiliar with but able to
learn about --- a known unknown. Then, determining how to extract the gradient and Hessian for
vector norms; in particular, the infinity norm. Another known unknown, one which he solved via
an analytic expression after much effort. However, it contained an unknown unknown: the
use of auto-differentiation and finite difference methods that would have eased and sped his
solution. The third, a trick involving first optimizing the ratio of a p-norm over the 2-norm where
$p>2$, then increasing $p$ iteratively until finally jumping to infinity. An unknown unknown:
the motivation behind this technique. Unless you've done a lot of work with solutions to polynomials, you likely
wouldn't recognize this as a form of homotopy continuation.

\subsection*{Academia}
The meta-problem of how one creates such an environment can also be addressed via diversity.
I was, effectively, a first-generation college student (both parents attended
some community college, but neither received a degree). I nearly crashed and burned. There
were so many unknown unknowns in navigating courses, academic culture, housing, financial aid...
I didn't even realize I should have been asking questions. The only thing that really turned
this around was an empathetic professor who took me under his wing and gave me the time no
others seemed able or willing to provide. How many students have fallen through that gap?
Furthermore, I am an otherwise a relatively privileged individual of society. Where am
I blind to the difficulties of those less fortunate than I? What power dynamics have I
been able to blithely ignore; worse yet, where have I possibly harmed another via my
ignorance?

\subsection*{Graduate Education}
Encouraging a diverse student body will lead to a broader pool of faculty that
have experienced these hardships, and know that there are students for whom these challenges
are a daily struggle. They needn't even have answers for these problems --- simply being
aware enough to ask shifts them from unknown unknowns to known unknowns, and from there
we can work to solve them. My intent in a graduate program is to address both
facets by immersing myself in an environment that counters both problems, and
then to eventually be in such a position as to actively work to change other
institutions.

\end{document}
