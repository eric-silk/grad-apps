I have contributed to the diversity of higher eductation through one primary vector:
FIRST robotics. This program offers high school students the opportunity to design and
build a robot for a competitive game that changes every year. I mentored one such team
for approximately two years here in the Palouse. This team was noteworthy in two ways:
it had a relatively high concentration of young women on the team, and drew largely from
rural communities in the area surrounding Pullman, WA.

The prior group is obviously a focus of many efforts. We went to lengths to ensure
the young women on the team were in a comfortable, inclusive environment, and to speak
their minds when it came to design, building, and controlling the robot, along with
stepping into leadership roles. In particular, one freshman member came from a
homeschooling background with only one computer that was for ``things like bills''.
But, she had an interest in the programming team (which I mentored).
By the end of her first year, she was one of the more productive members on the
programming team, and when I last saw her she was the lead programmer for the team.
Most tangibly, she was planning to apply to CS programs at the local universities.
No mean feat to go from not having a solid grasp of the Start Menu to writing
custom C++.

Advocacy for latter group is perhaps less in vogue. As evidenced by the prior anecdote, some
students lived in locations without consistent internet, little tech in their
homes, and small, underfunded schools or homeschooling. It is difficult to motivate
higher education in such an environment, owing to lack of examples or weakness of
pedagogy. Our program drew students from under this umbrella and gave them concrete
applications working with local engineers and professors. To my knowledge, all
students who stuck with the program enrolled in four-year degree programs. In contrast,
some other non-particpating schools in the area have among the worst college enrollment
rates in the state.

Both groups critically need representation in higher eductation, especially in STEM.
More eloquent explanations of the gender divide than I can muster exist already, so
I'll neglect to expand upon it here. The latter is best framed by recognizing that
rural areas are facing fewer and fewer opportunities for upward mobility. This is
justification enough -- all people deserve these chances. To further sweeten things,
mitigating this opportunity gap can help mitigate the growing cultural gap between
urban and rural centers. One merely needs to look at recent electoral maps and
political sentiments to see this is a growing problem. Broader experiences through
the inclusion of underrepresented rural citizens would do much (on both sides of
the divide) to ameliorate these tensions. Finally, STEM is an exercise in formalized
creativity.  Broader backgrounds promote more productive creation. A prime example
is Tom Mueller, co-founder of Space-X, born and raised in St. Maries, Idaho
(pop. 2,402). Where would rocketry be had he not been proximal to the University of Idaho?
