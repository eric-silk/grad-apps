% !TEX program = xelatex
%%%%%%%%%%%%%%%%%%%%%%%%%%%%%%%%%%%%%%%%%%%%%%%%%%%%%%%%%%%%%%%%%%%%%
%% Title: SOP LaTeX Template
%% Author: Soonho Kong / soonhok@cs.cmu.edu
%% Modified by: Eric Silk
%% Created: 2012-11-12
%% Modified: 2021
%%%%%%%%%%%%%%%%%%%%%%%%%%%%%%%%%%%%%%%%%%%%%%%%%%%%%%%%%%%%%%%%%%%%%

%%%%%%%%%%%%%%%%%%%%%%%%%%%%%%%%%%%%%%%%%%%%%%%%%%%%%%%%%%%%%%%%%%%%%
%%
%% Requirement:
%%     You need to have the `Adobe Caslon Pro` font family.
%%     For more information, please visit:
%%     http://store1.adobe.com/cfusion/store/html/index.cfm?store=OLS-US&event=displayFontPackage&code=1712
%%
%% How to Compile:
%%     $ xelatex main.tex
%%
%%%%%%%%%%%%%%%%%%%%%%%%%%%%%%%%%%%%%%%%%%%%%%%%%%%%%%%%%%%%%%%%%%%%%

\documentclass[letterpaper]{article}
\usepackage[letterpaper,margin=1.0in,noheadfoot]{geometry}
\usepackage{fontspec, color, enumerate, sectsty}
\usepackage[normalem]{ulem}

%%%%%%%%%%%%%%%%%%%%%%%%%%%%%%%%%%%%%%%%%%%%%%%%%%%%%%%%%%%%%%%%%%%%%
%                      YOUR INFORMATION
%
%      PLEASE EDIT THE FOLLOWING LINES ACCORDINGLY!!
%%%%%%%%%%%%%%%%%%%%%%%%%%%%%%%%%%%%%%%%%%%%%%%%%%%%%%%%%%%%%%%%%%%%%
\newcommand{\soptitle}{Personal Statement}
\newcommand{\yourname}{Eric Silk}
\newcommand{\youremail}{esilk16@uw.edu}

%% FONTS SETUP
\defaultfontfeatures{Mapping=tex-text}
\usepackage[bookmarks, colorlinks, breaklinks,
pdftitle={\yourname - \soptitle},pdfauthor={\yourname}, unicode]{hyperref}
\hypersetup{linkcolor=magneta,citecolor=magenta,filecolor=magenta,urlcolor=[named]{WildStrawberry}}

%%%%%%%%%%%%%%%%%%%%%%%%%%%%%%%%%%%%%%%%%%%%%%%%%%%%%%%%%%%%%%%%%%%%%
%                      Title and Author Name
%%%%%%%%%%%%%%%%%%%%%%%%%%%%%%%%%%%%%%%%%%%%%%%%%%%%%%%%%%%%%%%%%%%%%
\begin{document}
\begin{center}{\huge \scshape \soptitle}\end{center}
\begin{center}\vspace{0.2em} {\Large \yourname\\}
  {\youremail\\}{Electrical and Computer Engineering}\end{center}
\frenchspacing
My motivation to pursue graduate education was simple: I was not satisfied
with many of the explanations I had been given in my undergraduate engineering education and
felt I had gaps --- known unknowns, and unknown unknowns.
An illustrative anecdote was when a professor was introducing phase plane plots for
transfer functions: he drew the two axes and indicated the left-hand side, ``If all
your system's poles are here, you will be stable.'' When I asked why, he made a face
as though he'd never been asked that, and told me to ``...try it in MATLAB, and you'll see.''

This pattern was repeated in several other courses, and then in my first role as an engineer.
Soon, however, I had the privilege of working alongside a pair of highly talented mathematicians
in my current role as a Research  Engineer in Schweitzer Engineering Laboratories' Infrastructure Defense division.
Their casual, almost
effortless understanding of many of the more advanced topics in signal processing, optimization,
and more was, I'll admit, a point of envy. I decided that while I'm generally fine with being
the least knowledgeable in a group, this was too much of a disparity. Another illustrative
anecdote: we were researching anomaly detection techniques. My one coworker came to my desk,
threw down an IEEE journal article and said (paraphrasing): ``Look what these fools did!''
They had, effectively, combined several statistical quantities without discernible connection
to one another and said ``if this quantity exceeds this oddly specific value, we
call it an anomaly.'' It was akin to saying ``if the ratio of pigeons to hawks on your
line raised to the overall line capacitance exceeds 48.23458, you have an event.''
The problem? I could not see that obvious error. Engineering practice is a house built on the
foundation of mathematics, and I realized my abode was built on soft clay.

I soon applied for a program in Applied Math at the University of Washington to address this.
In fact, my first course illuminated the left-hand plane mechanism (the homogenous response
is comprised of complex exponentials with the poles in the exponent --- of course
negative real portions lead to asymptotic stability) and why nonlinearity is needed for real
oscillators (What do you call Van der Pol's hammock? A relaxation oscillator!). That was just
the start.
This program has given me the tools and confidence to readily engage with mathematical literature
to tackle challenging engineering problems. My work in our Infrastructure Defense division has
underpinned the immense gravity of these problems. I now want to continue my education in this
space, to become the best engineer I can be, and to speak with confidence when it comes
to problems of grave importance. And, given the best possible
outcome, be able to answer questions from students, peers, and political leaders
with intuitive, rigorous reasoning, rather than deferring to MATLAB.


%%%%%%%%%%%%%%%%%%%%%%%%%%%%%%%%%%%%%%%%%%%%%%%%%%%%%%%%%%%%%%%%%%%%%
%                      SOP Body
% NOTE: Use \amper instead of \&
%%%%%%%%%%%%%%%%%%%%%%%%%%%%%%%%%%%%%%%%%%%%%%%%%%%%%%%%%%%%%%%%%%%%%
\end{document}
