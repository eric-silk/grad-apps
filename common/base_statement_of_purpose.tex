% !TEX program = xelatex
%%%%%%%%%%%%%%%%%%%%%%%%%%%%%%%%%%%%%%%%%%%%%%%%%%%%%%%%%%%%%%%%%%%%%
%% Title: SOP LaTeX Template
%% Author: Soonho Kong / soonhok@cs.cmu.edu
%% Modified by: Eric Silk
%% Created: 2012-11-12
%% Modified: 2021
%%%%%%%%%%%%%%%%%%%%%%%%%%%%%%%%%%%%%%%%%%%%%%%%%%%%%%%%%%%%%%%%%%%%%

%%%%%%%%%%%%%%%%%%%%%%%%%%%%%%%%%%%%%%%%%%%%%%%%%%%%%%%%%%%%%%%%%%%%%
%%
%% Requirement:
%%     You need to have the `Adobe Caslon Pro` font family.
%%     For more information, please visit:
%%     http://store1.adobe.com/cfusion/store/html/index.cfm?store=OLS-US&event=displayFontPackage&code=1712
%%
%% How to Compile:
%%     $ xelatex main.tex
%%
%%%%%%%%%%%%%%%%%%%%%%%%%%%%%%%%%%%%%%%%%%%%%%%%%%%%%%%%%%%%%%%%%%%%%

\documentclass[letterpaper]{article}
\usepackage[letterpaper,margin=1.0in,noheadfoot]{geometry}
\usepackage{fontspec, color, enumerate, sectsty}
\usepackage[normalem]{ulem}

%%%%%%%%%%%%%%%%%%%%%%%%%%%%%%%%%%%%%%%%%%%%%%%%%%%%%%%%%%%%%%%%%%%%%
%                      YOUR INFORMATION
%
%      PLEASE EDIT THE FOLLOWING LINES ACCORDINGLY!!
%%%%%%%%%%%%%%%%%%%%%%%%%%%%%%%%%%%%%%%%%%%%%%%%%%%%%%%%%%%%%%%%%%%%%
\newcommand{\soptitle}{Statement of Purpose}
\newcommand{\yourname}{Eric Silk}
\newcommand{\youremail}{esilk16@uw.edu}

%% FONTS SETUP
\defaultfontfeatures{Mapping=tex-text}
\usepackage[bookmarks, colorlinks, breaklinks,
pdftitle={\yourname - \soptitle},pdfauthor={\yourname}, unicode]{hyperref}
\hypersetup{linkcolor=magneta,citecolor=magenta,filecolor=magenta,urlcolor=[named]{WildStrawberry}}

%%%%%%%%%%%%%%%%%%%%%%%%%%%%%%%%%%%%%%%%%%%%%%%%%%%%%%%%%%%%%%%%%%%%%
%                      Title and Author Name
%%%%%%%%%%%%%%%%%%%%%%%%%%%%%%%%%%%%%%%%%%%%%%%%%%%%%%%%%%%%%%%%%%%%%
\begin{document}
\begin{center}{\huge \scshape \soptitle}\end{center}
\begin{center}\vspace{0.2em} {\Large \yourname\\}
  {\youremail}\end{center}

%%%%%%%%%%%%%%%%%%%%%%%%%%%%%%%%%%%%%%%%%%%%%%%%%%%%%%%%%%%%%%%%%%%%%
%                      SOP Body
% NOTE: Use \amper instead of \&
%%%%%%%%%%%%%%%%%%%%%%%%%%%%%%%%%%%%%%%%%%%%%%%%%%%%%%%%%%%%%%%%%%%%%
\section*{Motivation and Background}
I am a Research Engineer at Schweitzer Engineering Laboratories (SEL), and graduating masters
student of Applied Mathematics at the University of Washington with a Bachelors of Science
in Electrical Engineering from the University of Idaho. My employer's mission is
one I believe in: to make electric power safer, more reliable, and more economical. On a daily
basis I perform at the intersection of applied math, electrical engineering, and software
engineering to further this goal.

When I transferred to Government Services from R\&D, it became readily apparent there was
an immense need for people who were capable of bridging the gap between the abstract and
sometimes esoteric world of mathematics and real-world applications in engineering.
To address my deficincies in the prior, I elected to pursue graduate education. 
Three years later, even with the huge strides I have made, it is obvious my work has
just started. Earning a PhD will allow me to transition into an independent, highly
productive researcher and to speak with authority on topics related to electric power.

\section*{Relevant Experience}
\paragraph{Career Research}
I must preface this section: due to its nature, I'm not able to discuss specifics of
much of the work I do for my employer, and we are definitely not allowed to publish.
However, I can say we are recognized as experts
on power systems by various federal agencies and regularly provide research, analysis,
and product to this end. In particular, we in the Data Analytics group marry this expertise
with high performance computing, machine learning, statistics, and traditional applied
mathematics to achieve novel and challenging goals. Of those I can discuss, several stand out.

Earlier this year I submitted a patent application for a compression method intended
to help streamline the massive data output of SEL's flagship relay, the T400L/T401L -- an
18 bit, megahertz sampling relay for the detection of traveling wave events on
transmission lines. We have a vested interested in long-term data captures at this
fidelity, but the existing implentation was naive and had excessive storage requirements,
(three weeks on a four terabyte drive). As a project during Dr. Kutz's class at the UW
(AMATH 582), I took his theme of assuming structure to solve $Ax=b$ given only $b$ and
assumed an autoregressive model to more compactly represent these time series,
producing lossy compression. Then, the errors are stored in Golomb-Rice codes, producing
a reduction of 5.7 to 7.4 times less than the existing implementation. This was a
major achievement -- it allows for one year of data to be stored on a single 3.5"
drive, or streaming to offsite storage using less bandwidth than a 1080p Netflix stream.

My first assignment was developing an Internet
of Things (IoT) based sensor system to measure and store Synchrophasor data (GPS timestamped
frequency, voltage, and phase) on the cloud. This involved selecting a computing platform,
developing code to interact with a phasor measurement unit (PMU), robust logging and
system restart behavior, and code for Amazon Web Services' IoT, Lambda, and S3
services. Then, these devices were hardened to prevent tampering through
careful device configuration (SSH keys and passwords, disabling of WiFi, disabling
of default accounts, etc.) We succesfully deployed several devices that showed greater
resiliency than SEL's existing solution. These devices have performed uninterrupted for
several years now, and we are currently investigating expansion of this sensor network.


A more recent achievement was the synthesis of a massive schematic dataset intended to
bootstrap a related machine learning model we are working on. Over ten-thousand Eagle
schematics were generated, including user specified components such as passives
(resistors, etc.), active components (diodes, BJT's, etc.), transformers, integrated
circuits, and more. This also included bounding boxes, component identifiers and values,
rotations -- all without human intervention. I am currently in progress of expanding
this capability to include more CAD programs.

Finally, I have applied optimization in several impactful ways. The first was generating
nonlinear transmission line designs to achieve particular performance characteristics
(impulse response rise time, peak value, etc.) that would be intractable with traditional
engineering techniques. While this was not a particulary succesful project, it taught me two
valuable lessons: have frank discussions with customers about requirements and ensure
both parties have clear objectives to minimize iterations; and, while open-source
software is a boon, niche projects can suffer from poor stability, reliance on end-of-lifed
components, or even just incorrect behaviors. Similarly, another project involved
fitting high-order, nonlinear transformer parameters through genetic algorithms. This
was quite sucessful, and allowed me to learn that judicious use of penalties prevents
compuationally correct but physically nonsensical models. This tool is still in use
by our power engineering team. Finally, a recent achievement was the implementation of a
crest-factor minimization problem in PyTorch that meets current state of the art, but
will serve as a testbed for different optimizers (the paper in question uses
Levenberg-Marquardt) and hyperparameter searches for these alternate optimizers.
Furthermore, it will be an excellent framework for other optimization problems
we regularly encounter without having to determine gradients and Hessians by hand.

\paragraph{Independent Study and Scalable Second Order Optimizers}
As an online Masters student, I did not have the opportunity to write a traditional
thesis. Instead, I approached Dr. Andrew Lumsdaine about doing an independent
study course over the summer of 2020. Topics ranged from power system load flow
simulations (using PETSc), CUDA/Thrust implementations of several matrix
decompositions, and the intersection of graphs and linear algebra. Although there was
no research output for this period of performance, I learned an incredible amount
and made enough of an impression to be brought on helping with his Second Order Methods
for Scalable Optimization project. This project involved using a trick (detailed
in our preprint here: \textbf{TODO}) to reduce the memory and compuational requirements
of Newton, Quasi-Newton, and other second order methods for applications in machine
learning. Our git repository can be found here: \textbf{TODO}. 

\section*{Career Goals}
I see three non-orthogonal avenues in my future: professorship, industry, and civil service.
All three share a common goal: perform at the intersection of applied math and engineering
while providing relevant, timely, and understandable analysis and education to students,
peers, or customers.

I needn't tell the audience the appeal of the first. I've been priveleged to have had
excellent professors, and have seen who I would want to be in engaging with students. I want
to provide power engineers with the tools to step beyond traditional power engineering
analyses and to engage with cutting-edge applied mathematics techniques. I regularly see
brilliant power engineers failing to articulate problem statements or using heavy-handed
approches out of ignorance -- much of my optimization work has grown out of statements
such as ``I don't think this equation is solvable, so I'm not sure we can use this''  or
``Well, this system is highly nonlinear...''. Being a educator and researcher could do
much to alleviate this as our power system continues to grow in complexity. Should a
full time, tenured faculty position not be in the cards, I would seek out options for
adjunct or affiliate positions so I can continue to engage with students.

In industry, or in civil service (working for an agency or a national lab), I would continue
much in my same role if at all possible: providing exemplary work for government customers
with the intent of making the grid safer, cheaper, and more reliable. This would take the
form of consulting and providing tools and techniques for use by the federal government,
and sound advice for shaping policy.


\section*{Institution Selection}

\section*{Concluding Remarks}
I hope I've demonstrated a clear updward trajectory in my career and research. I believe
I'm on the cusp of being a strong and independent researcher; a PhD from your program
will be the keystone in my ability to contribute in a more impactful way to electrical
engineering and applied mathematics.
\end{document}
